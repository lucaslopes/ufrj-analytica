% !TEX root = article.tex

\documentclass{article}
\usepackage[utf8]{inputenc}
\usepackage[portuguese]{babel}
\usepackage{amsmath}
\usepackage{amsfonts}
\usepackage{amssymb}
\usepackage{graphicx}
\usepackage[colorlinks=true, allcolors=blue]{hyperref}

\title{Deslocamento de Gestantes entre Municípios no Brasil: Análise do Acesso ao Parto Hospitalar pelo SUS na Década de 2010}
\author{Lucas Lopes Felipe}
\date{}

\begin{document}

\maketitle

\begin{abstract}
Este artigo investiga o deslocamento de gestantes entre municípios no Brasil para acesso a serviços de parto hospitalar pelo Sistema Único de Saúde (SUS) na década de 2010. A pesquisa explora a relação entre indicadores socioeconômicos dos municípios e o déficit na oferta de atendimento ao parto hospitalar, utilizando técnicas de aprendizado de máquina e análise de regressão linear. O objetivo é identificar os principais fatores que contribuem para a necessidade de deslocamento das gestantes e propor estratégias para melhorar a cobertura e acessibilidade do SUS nesse contexto.
\end{abstract}

\section{Introdução}

O Sistema Único de Saúde (SUS) do Brasil é um sistema de saúde público que visa proporcionar acesso universal e igualitário aos serviços de saúde. No entanto, em relação ao parto hospitalar, muitas gestantes precisam se deslocar para outros municípios para receber atendimento adequado. Este estudo analisa o histórico de deslocamento de gestantes entre municípios no Brasil e busca identificar correlações entre indicadores socioeconômicos dos municípios e o déficit na oferta de serviços de parto hospitalar.

Para isso, utilizamos dados do Sistema de Informações Hospitalares (SIH) do SUS, que fornecem informações sobre o município de residência e de internação de todos os partos hospitalares ocorridos entre 2010 e 2019. Além disso, empregamos dados socioeconômicos dos municípios, obtidos por meio do \href{https://basedosdados.org/}{Base dos Dados} e sua \href{https://basedosdados.github.io/mais/api_reference_python/}{API Python}, com informações do \href{https://basedosdados.org/dataset/br-ipea-acesso-oportunidades?bdm_table=estatisticas_2019}{Projeto Acesso a Oportunidades}.

\subsection{Organização do Artigo}

O restante deste artigo está organizado da seguinte forma:

\begin{itemize}
    \item Seção 2: Metodologia - Apresenta a metodologia utilizada na análise dos dados e descrição das técnicas de aprendizado de máquina e regressão linear empregadas.
    \item Seção 3: Resultados - Apresenta os principais resultados obtidos, incluindo a análise de tendências de deslocamento e a identificação dos municípios que mais recebem gestantes de outras localidades.
    \item Seção 4: Discussão - Discute os resultados, abordando possíveis explicações para os padrões observados e propondo estratégias para melhorar a oferta e acessibilidade do SUS em relação ao parto hospitalar.
    \item Seção 5: Conclusão - Resume os principais achados do estudo, destaca suas implicações e sugere direções para futuras pesquisas.
    \item Seção 6: Referências - Lista as referências bibliográficas utilizadas no artigo.
    \item Apêndice (se necessário) - Apresenta informações complementares, como detalhes sobre a coleta e processamento dos dados, ou análises adicionais realizadas.
\end{itemize}

\end{document}
    
